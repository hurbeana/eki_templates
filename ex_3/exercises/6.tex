In 1713, Nicolas Bernoulli investigated a problem, nowadays referred to as the \textit{St. Petersburg paradox}, which works as follows. You have the opportunity to play a game in which a fair coin is tossed repeatedly until it comes up heads. If the first head appears on the $n$-th toss, you win $2^n$ Euros.

\begin{enumerate}[label=(\alph*)]
    \item Show that the expected monetary value of this game is not finite.
    \item Daniel Bernoulli, the cousin of Nicolas, resolved the apparent paradox in 1738 by suggesting that the utility of money is measured on a logarithmic scale, i.e., $U(S_n) = alog_2 n + b$, where $S_n (n > 0)$ is the state of having $n$ Euros and $a,b$ are constants. What is the expected utility of the game under this assumption? Assume, for simplicity, an initial wealth of $0$ Euros and that no stake has to be paid in order to play the game.
\end{enumerate}